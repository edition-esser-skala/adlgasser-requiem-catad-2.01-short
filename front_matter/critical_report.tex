\documentclass{ees}

\begin{document}

\eesTitlePage

\eesCriticalReport{
  – & –   & –     & Compared to authentic copies of the work
                    (e.g., A-KR D 40/2, RISM 600171007), \B1 contains
                    large cuts, which comprise the following bars:
                    \textit{Introitus} (bars 1–12 and 38–45, yielding 27 bars
                                       in \B1 vs 47 bars in the long version),
                    \textit{Te decet} (52–60, 77f, 106f, 113; 54 vs 68),
                    \textit{Kyrie} (6–8 (replaced by one new bar), 27–38;
                                   43 vs 29),
                    \textit{Dies iræ} (62–65; 66 vs 62),
                    \textit{Liber scriptus} (74–81, 146–155, 176–183 (1 new),
                                            200–207 (1 new); 254 vs 222),
                    \textit{Lacrymosa} (no cuts; 30 bars),
                    \textit{Domine Jesu Christe} (56–60 (1 new), 83–87 (1 new),
                                                 90–92 (2 new); 94 vs 85),
                    \textit{Quam olim Abrahæ 1} (replaced by the shorter repeat;
                                                136 vs 41),
                    \textit{Hostias} (239–249; 262–266, 273–284 (1 new);
                                     65 vs 38),
                    \textit{Quam olim Abrahæ 2} (no cuts; 41 bars),
                    \textit{Sanctus} (10–13 (1 new), 19–20 (1 new), 24–35
                                     (2 new), 39 (2nd half)–41 (1st half);
                                     48 vs 32),
                    \textit{Benedictus} (5–19 (1 new), 25–77, 106 (2nd half)–%
                                        108 (1st half); 115 vs 46),
                    \textit{Agnus Dei} (no cuts, 43 bars), and
                    \textit{Requiem} (same as \textit{Introitus}; 47 vs 27).
                    Moreover, ob and fag parts have likely been added
                    by later hand.\\
  1 & –   & –     & The \textit{da capo} of the \textit{Requiem} (bars 1–27)
                    is written out in \B1. \\
    & 25  & T     & bar in \B1: c′2. (also in the \textit{da capo}) \\
    & 26  & A     & 1st \quarterNote\ in \B1: g′4 \\
    & 35  & vl 2  & 3rd to 6th \eighthNote\ in \B1: 4 × f′8 \\
    & 45  & vla   & 1st \quarterNote\ in \B1: a+e′4 \\
    & 53  & vl 2  & 1st \eighthNote\ in \B1: grace g″16–f″8 \\
    & 76  & fag 2 & 3rd \quarterNote\ in \B1: \flat B4 \\
    & 78  & vl 1  & grace note missing in \B1 \\
  2 & 26  & ob 2  & bar missing in \B1 \\
    & 27  & S     & 2nd \eighthNote\ in \B1: d″8 \\
  3 & 26  & vl 2  & 1st \halfNote\ in \B1: 8 × a′16 \\
    & 38  & S     & grace note missing in \B1 \\
    & 42  & S     & 1st \quarterNote\ in \B1: d′4 \\
    & 92  & A     & 4th \eighthNote\ in \B1: g′8 \\
    & 121 & vl 2  & 2d \quarterNote\ in \B1 unison with vl 1 \\
    & 138 & trb 1 & 4th \sixteenthNote\ in \B1: \flat b′16 \\
    & 158 & vl 1  & grace note missing in \B1 \\
    & 274 & trb 2 & 2nd \quarterNote\ in \B1: e′4 \\
    & 287 & org   & 5th \eighthNote\ in \B1: \flat b8 \\
    & 298 & trb 2 & 8th \sixteenthNote\ in \B1: c′16 \\
    & 309 & trb 1 & 2nd \quarterNote\ in \B1: d′4 \\
  4 & –   & –     & The \textit{da capo} of the \textit{Quam olim}
                    (bars 86–128) is written out in \B1. \\
    & 79  & vl 2  & grace note missing in \B1 \\
  5 & 12  & trb 1 & 3rd \quarterNote\ in \B1: d′4 \\
    & 12  & trb 2 & 3rd \quarterNote\ in \B1: b4 \\
    & 25  & ob 1  & 5th \eighthNote\ missing in \B1 \\
  7 & 4   & vl 1  & 8th \thirtysecondNote\ in \B1: e′32 \\
    & 4   & B     & 1st \quarterNote\ in \B1: A4 \\
    & 16  & vl 1  & 2nd \quarterNote\ in \B1: 6 × d′+c″16 \\
    & 19  & T     & 7th \eighthNote\ in \B1: g+b8 \\
    & 38  & vla, org & 12th \sixteenthNote\ in \B1: c16 \\
}

\eesToc{}

\eesScore

\end{document}
